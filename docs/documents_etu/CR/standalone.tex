% ===================================================== %
% ===================  PREAMBULE ====================== %
% ===================================================== %

\documentclass[a4paper, french]{report}
\usepackage{config}

% ===================================================== %
% ===================================================== %
% ===================================================== %

\begin{document}

% Page de titre standarde
\begin{titlepage}
    \begin{flushleft}
        \includegraphics[width=5cm]{UP.png}\par
        \centering
        
        \vspace{13\baselineskip}       
        \HRule \\[0.4cm]

        {\Huge 
        Projet \textit{Interface d'un jeu d'exploration}\\ (Rapport)\par}
        \vspace{0.4cm}
        \HRule
        \vfill
      
        Auteur(s): Florian Legendre, Alexis Louail,\\Vincent Tourenne\medskip \par
        
        \includegraphics[scale=0.7]{creative_commons.png}\par
    \end{flushleft}
\end{titlepage}

% Explication des légendes et abbréviations Standard
\newpage
\begin{LARGE}
Légendes et Abbréviations utilisées\\\\\\\\
\end{LARGE}
\textbf{Question:} Ceci est une question de l'enseignant\\
\textbf{Réponse:} Ceci est une réponse de l'enseignant ou validée par l'enseignant\\
\textbf{Réponse:} \textit{Ceci est une réponse du ou d'un des auteurs non validée par l'enseignant}\\\\

\begin{lstlisting}[style=C, caption=Exemple de code source]
Ceci est du code source.
Selon les langages, différents mots seront colorés selon 
si ce sont des mots clefs ou non (comme int, char, etc.).
\end{lstlisting}

\begin{mdframed}[style=Bash]
\begin{lstlisting}[style=Bash, caption=Exemple d'une pseudo capture d'écran Bash]
Ceci est un formattage automatique Latex d'un texte copié-collé
directement depuis un terminal Bash ayant valeur de capture
d'écran. La coloration correspond à une coloration quelconque 
d'un terminal Bash (les chemins étant habituellement coloré et 
le nom de l'utilisateur aussi comme crex@crex:~$ ...)
\end{lstlisting}
\end{mdframed}

\begin{mdframed}[style=Bash]
\begin{lstlisting}[style=Bash_pedago, caption=Exemple d'une capture formaté en style 'Tutoriel de Commande Bash' ou TCB]
Ceci est un formattage automatique Latex d'un texte copié-collé
directement depuis un terminal Bash et dont la coloration a été
revisitée pour mettre en relief les commandes natives de Bash
comme cp en cyan, les alias ou scripts faits par l'utilisateur 
en jaune (Ex. mygcc), les options en magenta (comme -i ), les 
opérateurs de redirection en rouge (comme > | < ) et l'invite de
commande $ en vert.
\end{lstlisting}
\end{mdframed}

% Sommaire Standard
\newpage
\pagestyle{empty}
\tableofcontents
\addtocontents{toc}{\protect\thispagestyle{empty} 
                    \protect\pagestyle{empty}}
\pagestyle{plain}

\newpage



%%% ---------------------------------------------- %%%
%%% ------------------ Début TP ------------------ %%%
%%% ---------------------------------------------- %%%

\subfile{./content.tex}
\end{document}